
%% bare_jrnl.tex
%% V1.4b
%% 2015/08/26
%% by Michael Shell
%% see http://www.michaelshell.org/
%% for current contact information.
%%
%% This is a skeleton file demonstrating the use of IEEEtran.cls
%% (requires IEEEtran.cls version 1.8b or later) with an IEEE
%% journal paper.
%%
%% Support sites:
%% http://www.michaelshell.org/tex/ieeetran/
%% http://www.ctan.org/pkg/ieeetran
%% and
%% http://www.ieee.org/

%%*************************************************************************
%% Legal Notice:
%% This code is offered as-is without any warranty either expressed or
%% implied; without even the implied warranty of MERCHANTABILITY or
%% FITNESS FOR A PARTICULAR PURPOSE! 
%% User assumes all risk.
%% In no event shall the IEEE or any contributor to this code be liable for
%% any damages or losses, including, but not limited to, incidental,
%% consequential, or any other damages, resulting from the use or misuse
%% of any information contained here.
%%
%% All comments are the opinions of their respective authors and are not
%% necessarily endorsed by the IEEE.
%%
%% This work is distributed under the LaTeX Project Public License (LPPL)
%% ( http://www.latex-project.org/ ) version 1.3, and may be freely used,
%% distributed and modified. A copy of the LPPL, version 1.3, is included
%% in the base LaTeX documentation of all distributions of LaTeX released
%% 2003/12/01 or later.
%% Retain all contribution notices and credits.
%% ** Modified files should be clearly indicated as such, including  **
%% ** renaming them and changing author support contact information. **
%%*************************************************************************


% *** Authors should verify (and, if needed, correct) their LaTeX system  ***
% *** with the testflow diagnostic prior to trusting their LaTeX platform ***
% *** with production work. The IEEE's font choices and paper sizes can   ***
% *** trigger bugs that do not appear when using other class files.       ***                          ***
% The testflow support page is at:
% http://www.michaelshell.org/tex/testflow/



\documentclass[journal]{IEEEtran}
%
% If IEEEtran.cls has not been installed into the LaTeX system files,
% manually specify the path to it like:
% \documentclass[journal]{../sty/IEEEtran}





% Some very useful LaTeX packages include:
% (uncomment the ones you want to load)


% *** MISC UTILITY PACKAGES ***
%
%\usepackage{ifpdf}
% Heiko Oberdiek's ifpdf.sty is very useful if you need conditional
% compilation based on whether the output is pdf or dvi.
% usage:
% \ifpdf
%   % pdf code
% \else
%   % dvi code
% \fi
% The latest version of ifpdf.sty can be obtained from:
% http://www.ctan.org/pkg/ifpdf
% Also, note that IEEEtran.cls V1.7 and later provides a builtin
% \ifCLASSINFOpdf conditional that works the same way.
% When switching from latex to pdflatex and vice-versa, the compiler may
% have to be run twice to clear warning/error messages.





% *** CITATION PACKAGES ***
%
\usepackage{cite}
% cite.sty was written by Donald Arseneau
% V1.6 and later of IEEEtran pre-defines the format of the cite.sty package
% \cite{} output to follow that of the IEEE. Loading the cite package will
% result in citation numbers being automatically sorted and properly
% "compressed/ranged". e.g., [1], [9], [2], [7], [5], [6] without using
% cite.sty will become [1], [2], [5]--[7], [9] using cite.sty. cite.sty's
% \cite will automatically add leading space, if needed. Use cite.sty's
% noadjust option (cite.sty V3.8 and later) if you want to turn this off
% such as if a citation ever needs to be enclosed in parenthesis.
% cite.sty is already installed on most LaTeX systems. Be sure and use
% version 5.0 (2009-03-20) and later if using hyperref.sty.
% The latest version can be obtained at:
% http://www.ctan.org/pkg/cite
% The documentation is contained in the cite.sty file itself.



% \usepackage{graphicx}
% \usepackage{subfigure}
% \usepackage{threeparttable}
% \usepackage{enumitem}
% \usepackage{bigstrut,multirow}
% \usepackage{algorithm}  
% \usepackage{algorithmicx}  
% \usepackage{algpseudocode}  
% \usepackage{amsmath} 
% \usepackage{lipsum}
% \usepackage{cleveref}



% *** GRAPHICS RELATED PACKAGES ***
%
% \ifCLASSINFOpdf
\usepackage[pdftex]{graphicx}
  % declare the path(s) where your graphic files are
  % \graphicspath{{../pdf/}{../jpeg/}}
  % and their extensions so you won't have to specify these with
  % every instance of \includegraphics
  % \DeclareGraphicsExtensions{.pdf,.jpeg,.png}
% \else
  % or other class option (dvipsone, dvipdf, if not using dvips). graphicx
  % will default to the driver specified in the system graphics.cfg if no
  % driver is specified.
  % \usepackage[dvips]{graphicx}
  % declare the path(s) where your graphic files are
  % \graphicspath{{../eps/}}
  % and their extensions so you won't have to specify these with
  % every instance of \includegraphics
  % \DeclareGraphicsExtensions{.eps}
% \fi
% graphicx was written by David Carlisle and Sebastian Rahtz. It is
% required if you want graphics, photos, etc. graphicx.sty is already
% installed on most LaTeX systems. The latest version and documentation
% can be obtained at: 
% http://www.ctan.org/pkg/graphicx
% Another good source of documentation is "Using Imported Graphics in
% LaTeX2e" by Keith Reckdahl which can be found at:
% http://www.ctan.org/pkg/epslatex
%
% latex, and pdflatex in dvi mode, support graphics in encapsulated
% postscript (.eps) format. pdflatex in pdf mode supports graphics
% in .pdf, .jpeg, .png and .mps (metapost) formats. Users should ensure
% that all non-photo figures use a vector format (.eps, .pdf, .mps) and
% not a bitmapped formats (.jpeg, .png). The IEEE frowns on bitmapped formats
% which can result in "jaggedy"/blurry rendering of lines and letters as
% well as large increases in file sizes.
%
% You can find documentation about the pdfTeX application at:
% http://www.tug.org/applications/pdftex





% *** MATH PACKAGES ***
%
\usepackage{amsmath}
% A popular package from the American Mathematical Society that provides
% many useful and powerful commands for dealing with mathematics.
%
% Note that the amsmath package sets \interdisplaylinepenalty to 10000
% thus preventing page breaks from occurring within multiline equations. Use:
%\interdisplaylinepenalty=2500
% after loading amsmath to restore such page breaks as IEEEtran.cls normally
% does. amsmath.sty is already installed on most LaTeX systems. The latest
% version and documentation can be obtained at:
% http://www.ctan.org/pkg/amsmath





% *** SPECIALIZED LIST PACKAGES ***
%
\usepackage{algorithmic}
% algorithmic.sty was written by Peter Williams and Rogerio Brito.
% This package provides an algorithmic environment fo describing algorithms.
% You can use the algorithmic environment in-text or within a figure
% environment to provide for a floating algorithm. Do NOT use the algorithm
% floating environment provided by algorithm.sty (by the same authors) or
% algorithm2e.sty (by Christophe Fiorio) as the IEEE does not use dedicated
% algorithm float types and packages that provide these will not provide
% correct IEEE style captions. The latest version and documentation of
% algorithmic.sty can be obtained at:
% http://www.ctan.org/pkg/algorithms
% Also of interest may be the (relatively newer and more customizable)
% algorithmicx.sty package by Szasz Janos:
% http://www.ctan.org/pkg/algorithmicx




% *** ALIGNMENT PACKAGES ***
%
\usepackage{array}
% Frank Mittelbach's and David Carlisle's array.sty patches and improves
% the standard LaTeX2e array and tabular environments to provide better
% appearance and additional user controls. As the default LaTeX2e table
% generation code is lacking to the point of almost being broken with
% respect to the quality of the end results, all users are strongly
% advised to use an enhanced (at the very least that provided by array.sty)
% set of table tools. array.sty is already installed on most systems. The
% latest version and documentation can be obtained at:
% http://www.ctan.org/pkg/array


% IEEEtran contains the IEEEeqnarray family of commands that can be used to
% generate multiline equations as well as matrices, tables, etc., of high
% quality.




% *** SUBFIGURE PACKAGES ***
%\ifCLASSOPTIONcompsoc
% \usepackage[caption=false,font=normalsize,labelfont=sf,textfont=sf]{subfig}
%\else
\usepackage[caption=false,font=footnotesize]{subfig}
%\fi
% subfig.sty, written by Steven Douglas Cochran, is the modern replacement
% for subfigure.sty, the latter of which is no longer maintained and is
% incompatible with some LaTeX packages including fixltx2e. However,
% subfig.sty requires and automatically loads Axel Sommerfeldt's caption.sty
% which will override IEEEtran.cls' handling of captions and this will result
% in non-IEEE style figure/table captions. To prevent this problem, be sure
% and invoke subfig.sty's "caption=false" package option (available since
% subfig.sty version 1.3, 2005/06/28) as this is will preserve IEEEtran.cls
% handling of captions.
% Note that the Computer Society format requires a larger sans serif font
% than the serif footnote size font used in traditional IEEE formatting
% and thus the need to invoke different subfig.sty package options depending
% on whether compsoc mode has been enabled.
%
% The latest version and documentation of subfig.sty can be obtained at:
% http://www.ctan.org/pkg/subfig




% *** FLOAT PACKAGES ***
%
\usepackage{fixltx2e}
% fixltx2e, the successor to the earlier fix2col.sty, was written by
% Frank Mittelbach and David Carlisle. This package corrects a few problems
% in the LaTeX2e kernel, the most notable of which is that in current
% LaTeX2e releases, the ordering of single and double column floats is not
% guaranteed to be preserved. Thus, an unpatched LaTeX2e can allow a
% single column figure to be placed prior to an earlier double column
% figure.
% Be aware that LaTeX2e kernels dated 2015 and later have fixltx2e.sty's
% corrections already built into the system in which case a warning will
% be issued if an attempt is made to load fixltx2e.sty as it is no longer
% needed.
% The latest version and documentation can be found at:
% http://www.ctan.org/pkg/fixltx2e


\usepackage{stfloats}
% stfloats.sty was written by Sigitas Tolusis. This package gives LaTeX2e
% the ability to do double column floats at the bottom of the page as well
% as the top. (e.g., "\begin{figure*}[!b]" is not normally possible in
% LaTeX2e). It also provides a command:
%\fnbelowfloat
% to enable the placement of footnotes below bottom floats (the standard
% LaTeX2e kernel puts them above bottom floats). This is an invasive package
% which rewrites many portions of the LaTeX2e float routines. It may not work
% with other packages that modify the LaTeX2e float routines. The latest
% version and documentation can be obtained at:
% http://www.ctan.org/pkg/stfloats
% Do not use the stfloats baselinefloat ability as the IEEE does not allow
% \baselineskip to stretch. Authors submitting work to the IEEE should note
% that the IEEE rarely uses double column equations and that authors should try
% to avoid such use. Do not be tempted to use the cuted.sty or midfloat.sty
% packages (also by Sigitas Tolusis) as the IEEE does not format its papers in
% such ways.
% Do not attempt to use stfloats with fixltx2e as they are incompatible.
% Instead, use Morten Hogholm'a dblfloatfix which combines the features
% of both fixltx2e and stfloats:
%
% \usepackage{dblfloatfix}
% The latest version can be found at:
% http://www.ctan.org/pkg/dblfloatfix

\usepackage{cleveref}
\crefname{figure}{fig.}{fig.}

%\ifCLASSOPTIONcaptionsoff
%  \usepackage[nomarkers]{endfloat}
% \let\MYoriglatexcaption\caption
% \renewcommand{\caption}[2][\relax]{\MYoriglatexcaption[#2]{#2}}
%\fi
% endfloat.sty was written by James Darrell McCauley, Jeff Goldberg and 
% Axel Sommerfeldt. This package may be useful when used in conjunction with 
% IEEEtran.cls'  captionsoff option. Some IEEE journals/societies require that
% submissions have lists of figures/tables at the end of the paper and that
% figures/tables without any captions are placed on a page by themselves at
% the end of the document. If needed, the draftcls IEEEtran class option or
% \CLASSINPUTbaselinestretch interface can be used to increase the line
% spacing as well. Be sure and use the nomarkers option of endfloat to
% prevent endfloat from "marking" where the figures would have been placed
% in the text. The two hack lines of code above are a slight modification of
% that suggested by in the endfloat docs (section 8.4.1) to ensure that
% the full captions always appear in the list of figures/tables - even if
% the user used the short optional argument of \caption[]{}.
% IEEE papers do not typically make use of \caption[]'s optional argument,
% so this should not be an issue. A similar trick can be used to disable
% captions of packages such as subfig.sty that lack options to turn off
% the subcaptions:
% For subfig.sty:
% \let\MYorigsubfloat\subfloat
% \renewcommand{\subfloat}[2][\relax]{\MYorigsubfloat[]{#2}}
% However, the above trick will not work if both optional arguments of
% the \subfloat command are used. Furthermore, there needs to be a
% description of each subfigure *somewhere* and endfloat does not add
% subfigure captions to its list of figures. Thus, the best approach is to
% avoid the use of subfigure captions (many IEEE journals avoid them anyway)
% and instead reference/explain all the subfigures within the main caption.
% The latest version of endfloat.sty and its documentation can obtained at:
% http://www.ctan.org/pkg/endfloat
%
% The IEEEtran \ifCLASSOPTIONcaptionsoff conditional can also be used
% later in the document, say, to conditionally put the References on a 
% page by themselves.




% *** PDF, URL AND HYPERLINK PACKAGES ***
%
\usepackage{url}
% url.sty was written by Donald Arseneau. It provides better support for
% handling and breaking URLs. url.sty is already installed on most LaTeX
% systems. The latest version and documentation can be obtained at:
% http://www.ctan.org/pkg/url
% Basically, \url{my_url_here}.


\usepackage{color}
\usepackage{multirow}
\usepackage{amsfonts}
\usepackage{threeparttable}
\usepackage{bigstrut}
% *** Do not adjust lengths that control margins, column widths, etc. ***
% *** Do not use packages that alter fonts (such as pslatex).         ***
% There should be no need to do such things with IEEEtran.cls V1.6 and later.
% (Unless specifically asked to do so by the journal or conference you plan
% to submit to, of course. )


% correct bad hyphenation here
\hyphenation{op-tical net-works semi-conduc-tor}


\begin{document}
%
% paper title
% Titles are generally capitalized except for words such as a, an, and, as,
% at, but, by, for, in, nor, of, on, or, the, to and up, which are usually
% not capitalized unless they are the first or last word of the title.
% Linebreaks \\ can be used within to get better formatting as desired.
% Do not put math or special symbols in the title.
\title{INCAME : Interruptible CNN Accelerator for Multi-robot Exploration}
%
%
% author names and IEEE memberships
% note positions of commas and nonbreaking spaces ( ~ ) LaTeX will not break
% a structure at a ~ so this keeps an author's name from being broken across
% two lines.
% use \thanks{} to gain access to the first footnote area
% a separate \thanks must be used for each paragraph as LaTeX2e's \thanks
% was not built to handle multiple paragraphs
%

% \author{Michael~Shell,~\IEEEmembership{Member,~IEEE,}
%         John~Doe,~\IEEEmembership{Fellow,~OSA,}
%         and~Jane~Doe,~\IEEEmembership{Life~Fellow,~IEEE}% <-this % stops a space
% \thanks{M. Shell was with the Department
% of Electrical and Computer Engineering, Georgia Institute of Technology, Atlanta,
% GA, 30332 USA e-mail: (see http://www.michaelshell.org/contact.html).}% <-this % stops a space
% \thanks{J. Doe and J. Doe are with Anonymous University.}% <-this % stops a space
% \thanks{Manuscript received April 19, 2005; revised August 26, 2015.}}


\author{Jincheng Yu,
        Zhilin Xu,
        Shulin Zeng,
        Chao Yu,
        Jiantao Qiu,
        Chaoyang Shen,
        Yuanfan Xu, \\
        Guohao Dai,
        Yu Wang,
        and Huazhong Yang % <-this % stops a space
\thanks{This work is supported by National Key Research and Development Program of China (No. 2017YFA0207600 ), National Natural Science Foundation of China (No. U19B2019, 61832007 ), Tsinghua EE Xilinx AI Research Fund, Beijing National Research Center for Information Science and Technology (BNRist), and Beijing Innovation Center for Future Chips.
}% <-this % stops a space
\thanks{All authors are from Department of Electronic Engineering, Tsinghua University, Beijing, China.}% <-this % stops a space
\thanks{Email: yjc16@mails.tsinghua.edu.cn; yu-wang@mail.tsinghua.edu.cn} }

% note the % following the last \IEEEmembership and also \thanks - 
% these prevent an unwanted space from occurring between the last author name
% and the end of the author line. i.e., if you had this:
% 
% \author{....lastname \thanks{...} \thanks{...} }
%                     ^------------^------------^----Do not want these spaces!
%
% a space would be appended to the last name and could cause every name on that
% line to be shifted left slightly. This is one of those "LaTeX things". For
% instance, "\textbf{A} \textbf{B}" will typeset as "A B" not "AB". To get
% "AB" then you have to do: "\textbf{A}\textbf{B}"
% \thanks is no different in this regard, so shield the last } of each \thanks
% that ends a line with a % and do not let a space in before the next \thanks.
% Spaces after \IEEEmembership other than the last one are OK (and needed) as
% you are supposed to have spaces between the names. For what it is worth,
% this is a minor point as most people would not even notice if the said evil
% space somehow managed to creep in.



% The paper headers
% \markboth{Journal of \LaTeX\ Class Files,~Vol.~14, No.~8, August~2015}%
% {Shell \MakeLowercase{\textit{et al.}}: Bare Demo of IEEEtran.cls for IEEE Journals}
% The only time the second header will appear is for the odd numbered pages
% after the title page when using the twoside option.
% 
% *** Note that you probably will NOT want to include the author's ***
% *** name in the headers of peer review papers.                   ***
% You can use \ifCLASSOPTIONpeerreview for conditional compilation here if
% you desire.




% If you want to put a publisher's ID mark on the page you can do it like
% this:
%\IEEEpubid{0000--0000/00\$00.00~\copyright~2015 IEEE}
% Remember, if you use this you must call \IEEEpubidadjcol in the second
% column for its text to clear the IEEEpubid mark.



% use for special paper notices
%\IEEEspecialpapernotice{(Invited Paper)}


% make the title area
\maketitle

% As a general rule, do not put math, special symbols or citations
% in the abstract or keywords.
\begin{abstract}
\input{src/abstract.tex}
\end{abstract}

% Note that keywords are not normally used for peerreview papers.
% \begin{IEEEkeywords}
% IEEE, IEEEtran, journal, \LaTeX, paper, template.
% \end{IEEEkeywords}






% For peer review papers, you can put extra information on the cover
% page as needed:
% \ifCLASSOPTIONpeerreview
% \begin{center} \bfseries EDICS Category: 3-BBND \end{center}
% \fi
%
% For peerreview papers, this IEEEtran command inserts a page break and
% creates the second title. It will be ignored for other modes.
\IEEEpeerreviewmaketitle


\section{INTRODUCTION}
\input{src/intro.tex}

\section{RELATED WORK}
\input{src/relate.tex}

\section{INCAME Framework}
\input{src/incame.tex}

\section{Virtual-instruction-based Accelerator Interrupt}
\label{sec:cnninterrupt}
% The idea of interruption is introduced for dynamic multi-task scheduling. This section details the implementation of our \textbf{Virtual instruction Interruption}. \Cref{fig:interDPR} illustrates the idea of interruption to full utilize the hardware resources.


% \begin{figure*}[t]
% 	\centering
% 	\includegraphics[width=0.99\linewidth]{fig/interDPR.eps}
% 	\caption{Interruption to solve the hardware resources conflicts. When a high-priority task (FE) is started before the low-priority task (PR) is completed, the CNN accelerator backs up the status of PR to memory, and processes the FE task. When the high-priority task is completed, the low-priority task resumes and continues.
% }
% 	\label{fig:interDPR}
% \end{figure*}


\begin{table*}[t]
	\caption{Description for the basic instructions.}
	\footnotesize
	\centering
% Table generated by Excel2LaTeX from sheet 'Sheet3'
%\linespread{1.1}\selectfont
\begin{tabular}{|p{2.7em}|m{3.7em}|m{14.2em}|m{4.2em}<{\centering}|m{4.2em}<{\centering}|m{4em}<{\centering}|m{4em}<{\centering}||m{6em}<{\centering}|m{6em}<{\centering}|}
	\hline
	\multicolumn{1}{|c|}{Category} & \multicolumn{1}{c|}{Type} & \multicolumn{1}{c|}{Description} & \multicolumn{1}{c|}{Address 1} & \multicolumn{1}{c|}{Address 2} & \multicolumn{1}{c|}{Address 3} & \multicolumn{1}{c||}{Workload} & \multicolumn{1}{c|}{Backup} & \multicolumn{1}{c|}{Recovery} \\
	\hline
	\multirow{2}[4]{*}{LOAD} & LOAD\_W & Load weights/bias from DDR to on chip weight buffer. & Off-chip Addr & Weights-buffer Addr & - & Data Length & - & Weight / Inputdata \\
	\cline{2-9}\multicolumn{1}{|c|}{} & LOAD\_D & Load input data from DDR to on-chip data buffer. & Off-chip Addr & Data-buffer Addr & - & Data Length & - & Weight / Inputdata \\
	\hline
	\multirow{2}[4]{*}{CALC} & CALC\_I & Calculate intermediate results (from partial input channels) for some output channels from partial input channels. & Input Data Addr & Intermediate Data Addr & Weight Addr & Calc Size & Previous final results / Intermediate data & Weight / Inputdata / Intermediate data \\
	\cline{2-9}\multicolumn{1}{|c|}{} & CALC\_F & Calculate the results for some output channels from all input channels. The pooling, bias-adding and element-wise operations are operated in this instructions. & Input Data Addr & Output Data Addr & Weight Addr & Calc Size & Final results & Inputdata \\
	\hline
	SAVE & SAVE & Save the results from on-chip data buffer to DDR. & Off-chip Addr & Data-buffer Addr & - & Data Length & - & Inputdata \\
	\hline
	\end{tabular}%
	
	\label{tab:instr}%
 \end{table*}%


\begin{figure}[t]
	\centering
 % \vspace{-0.1cm} 
 % \setlength{\abovecaptionskip}{0cm} 
 % \setlength{\belowcaptionskip}{-0.6cm} 
	\includegraphics[width=0.99\linewidth]{fig/instructions.pdf}
	\vspace{-6mm}
	\caption{Fig(a), Original ISA. Fig(b), Virtual-Instruction ISA (VI-ISA).}
	\label{fig:instructions}
\end{figure}

In this section, we introduce our \textbf{virtual-instruction-based} method (\textit{VI} method) to enable accelerator interrupt. 
\textcolor{blue}{
Compared with the CPU-Like and Layer-by-Layer interrupt method, virtual-instruction-based method enables interrupt inside each layer to avoid high interrupt response latency and only transfers the minimum amount of data for backup and restore, to lower the extra cost of the interrupt.
}


% In this section, we introduce our virtual-instruction-based method to enable accelerator interrupt. To minimize the interrupt response latency and extra cost for interrupt, we propose the virtual-instruction-based accelerator interrupt method.

\subsection{ Instruction Driven Accelerator }
\label{sec:instrAcc}
There are three categories of instruction in the instruction-driven accelerator: LOAD, CALC, and SAVE~\cite{guo2017angel,qiu2016going,yu2018instruction}. The instruction description of each kind of instruction is listed in \Cref{fig:instructions}(a) and \Cref{tab:instr}.

The LOAD instruction moves input featuremaps and weights from DDR to on-chip memory. The SAVE instruction moves the calculated output features from on-chip memory to DDR. 

Each CALC instruction, including CALC\_I and CALC\_F, processes the convolution according to the hardware parallelism with $P_{h}$ lines from $ P_{in} $ input channels to $ P_{out}$ output channels. $P_{h}$, $ P_{in} $, and $ P_{out} $ are the parallelism along the height, input channel and output channel dimensions, which is determined by the hardware and original ISA.

\Cref{fig:singlesave}(a) illustrates the operation of CALC instructions. The convolution of the last $ P_{in} $ input channels is CALC\_F, and the convolutions for the former input channels are CALC\_I. The CALC\_F and the CALC\_I instructions for the same output channels, as well as the LOAD instructions for corresponding input featuremaps and weights, are considered as a \textbf{CalcBlob} (\Cref{sec:exampleVirtual}(c) lists an example for CalcBlob). In each CalcBlob, there is a LOAD\_W instruction for the corresponding weights. 
\textcolor{blue}{
However, when there are only few input channels (such $Ch_{in} = P_{in}$), all input data required by a CalcBlob can be fetched to the chip by a single LOAD\_D. 
During the CalbBlob execution, the input buffer remains unchanged. 
For the next CalcBlob, since the required input data is exactly the same, there is no need to execute a LOAD\_D.
Thus, some CalcBlobs do not have LOAD\_D instruction.
}


% Each CALC instruction, including CALC\_I and CALC\_F processes the convolution according to the hardware parallelism.
% Each CALC instruction, including CALC\_I and CALC\_F processes the convolution from input feature of the hardware input parallelism ($P_{in}$) to the output feature of the hardware output parallelism ($ P_{out}$), as illustrated in \Cref{fig:singlesave}(a). The convolution of the last input channels is CALC\_F, and the convolutions for the former input channels are CALC\_I. The CALC\_F and the CALC\_I instructions to generate the output channels, as well as the LOAD instructions for corresponding input featuremaps and weights are considered as a \textbf{CalcBlob}. Besides the convolution, some other operations like pooling is also represented in the CALC\_F instruction.

% Considerring to the limited on-chip data memory, the on-chip data buffer may not able to store all of the input and output featuremaps. To solve this problem, a CALC instruction is not designed for the entile featuremap, yet servel lines of the featuremap. The parallelism along the height dimension of a CALC instruction is denoted as 





% \subsection{Accelerator Interrupt }

\subsection{How To Interrupt: Virtual Instruction}
\label{sec:howinter}

As illustrated in \Cref{fig:singlesave}(e),there are four stages to handle interrupt, including: (1) Time for finishing the current operation, $t1$. (2) Time to backup, $t2$. (3) Time for the high-priority task, $t3$. (4) Time to restore the low-priority task ,$t4$. 
The latency to respond the interrupt is:
\begin{equation}
	t_{latency} = t_1+t_2
	\label{equ:latency}
\end{equation}
The extra cost for interrupt is: 
\begin{equation}
	t_{cost}=t_2+t_4
	\label{equ:cost}
\end{equation}	
For the instruction flow illustrated in \Cref{fig:singlesave}(c), the interrupt stages are shown in \Cref{fig:singlesave}(e).
There are different methods to implement interrupt in CNN accelerators.

\textbf{CPU-Like.}
When an interrupt request occurs in CPU, CPU backs up all the on-chip registers to DDR. However, there are only tens of registers in CPU, and the volume of the backed-up data is less than 1 KB~\cite{furber2000arm}. In CNN accelerators, there are hundreds of KB $\sim$ several MB on-chip caches~\cite{qiu2016going, guo2017angel} to store input featuremaps or weights. 
% If all on-chip caches are backed-up/recovered, the cost of data transfer in the accelerator is much higher than that of CPU. 
Thus, the extra data transfer increases both the interrupt response latency ($t_{latency}$) and the additional cost ($t_{cost}$).

\textbf{Layer-by-Layer.}
Most accelerators run the CNN layer by layer~\cite{qiu2016going,guo2017angel}. 
There is no extra data transfer for the accelerator to switch between different tasks after each layer, thus, $t_{cost}=0$. 
However, the position of the interrupt request is irregular and unpredictable. When an interrupt occurs inside a CNN layer, the CNN accelerator needs to finish the whole layer before switching, which leads to the high response latency ($t_{latency}$).

% The latency to respond the interrupt and the performance degradation of the CPU-like interrupt and Layer-by-Layer method will be evaluated in \Cref{sec:experiments}.



We propose the \textbf{virtual-instruction-based} method (\textit{VI} method) to enable low-latency interrupt. 
% Different from the CPU-like interrupt, which backup/recovery all the on-chip caches, only the on-chip cache which is still needed in future execution will be backed-up and restored. So that the amount of data transfer is much lower than that of CPU-like interrupt.
To reduce the interrupt response latency, our virtual-instruction-based method is interruptible inside each layer. We add some virtual instructions to the original instruction sequence to enable the interrupt.
The virtual instructions, which contain the backup and recovery instructions, are responsible for backing up and restoring on-chip caches. 

\textbf{Virtual SAVE} instructions back up the intermediate results from partial input channels or the final output results. There is no need to back up the input featuremaps and weights, because these inputs are already stored in DDR. 

\textbf{Virtual LOAD} instructions restore the input featuremaps from DDR to on-chip caches.
% because input featuremaps are loaded by one CalcBlob, and shared across subsequent CalcBlobs, and thus the subsequent CalcBlobs do not read the input featuremaps. 
Virtual LOAD instructions also need to restore the intermediate results from partial input channels backed up by the virtual SAVE instructions.

% By adding the virtual instructions, the CNN can be interrupted anywhere, and the latency to respond interrupt is reduced.


% For backup virtual instructions, the corresponding input data and weights are already stored in DDR. 
% Thus, there is no need to back up the input buffer and weight buffer. Only the intermediate data and the final output results are needed to be backed-up. 

% For recovery virtual instructions, the weights and input data, as well as the backed-up intermediate data, are needed to be restored from DDR to the on-chip cache.
% The accelerator can switch to a different task after the backup virtual instructions, and resume the execution by the recovery virtual instructions.

% By adding the virtual instructions, the CNN can be interrupted anywhere, and the latency to to response the interrupt is reduced. However, there virtual instructions are only valid when interrupt occurs. So we add a field in the origin instruction set, that indicates whether the instruction a virtual instruction. If no interrupt occurs, virtual instructions will be skipped and discarded, which can ensure the efficiency of uninterrupted execution. The modifications to the instruction set will be introduced in \Cref{sec:virtualinstr}. 


% However, the CPU-like interrupt would back up all the on-chip registers to DDR. In CPU, there are tens of registers, and the backed-up data is around 1 KB. In CNN accelerators, there are hundreds of KB~ several MB on-chip cache~\cite{qiu2016going, yu2018instruction}. If all the on-chip cache is backed-up and recovered, the cost of data transfer in the CNN accelerator is much higher than that of CPU.

% We propose the \textbf{virtual-instruction-based} method to enable low-latency interrupt. The low-priority task maintains the executing status itself, rather than the hardware or the interrupt handler used in CPU. Only the cache which is still needed in future execution will be backed-up and restored.

% The virtual instructions, which contain the backup and recovery instructions, are generated in the compilation phase, together with the normal instructions. 
% For backup instructions, the corresponding input data and weights are still stored in DDR. 
% There is no need to back up the input buffer and weight buffer, and only the intermediate data and the final output results are needed to be backed-up. 
% For recovery instructions, the weights and input data for future calculation, as well as the backed-up intermediate data, are needed to be restored from DDR to the on-chip cache.

% There is a field in the instruction set, that indicates whether the instruction a virtual instruction. If no interrupt occurs, virtual instructions will be skipped and discarded, which can ensure the efficiency of uninterrupted execution.







\begin{figure*}[t]
 % \flushleft
 \centering
 % \vspace{-0.1cm} 
 % \setlength{\abovecaptionskip}{0cm} 
 % \setlength{\belowcaptionskip}{-0.05cm} 
	\includegraphics[width=0.99\textwidth]{fig/singlesave.pdf} 	
	\vspace{-1mm} 
 \caption{
		Illustration of scheduling on the CNN accelerator. Fig(a), a CalcBlob is the instructions that calculate the final results of $P_{out}$ output channels from all input channels (the set of solid arrows). Fig(b), one SAVE instruction is only responsible for saving the results of one CalcBlob. Fig(c), one SAVE instruction is responsible for saving the results of several CalcBlobs. Fig(d) and Fig(e) illustrate the accelerator interrupt of Fig(b) and Fig(c). The latency ($t_{latency}$) and extra cost ($t_{cost}$) are labelled in Fig(e).
 }
	\label{fig:singlesave}
\end{figure*}



\subsection{ Where To Interrupt: After SAVE/CALC\_F }
\label{sec:whereinter}
The virtual-instruction-based method has two potential factors that may lead to system performance degradation: 1) The extra data transfer to backup/restore running status takes up additional bandwidth resources. 2) The instruction fetching for the virtual instructions also uses bandwidth resources.
% Even they are skipped and discarded.
To address the above problems of virtual-instruction-based method, we analyze the interrupt cost and select the positions of adding the virtual instructions.
The backup/recovery data for different interrupt positions at each kind of instruction are listed in the Backup/Recovery columns of \Cref{tab:instr}. The backup/recovery data transfer for each instruction is analyzed as follows:

\textbf{LOAD\_W / LOAD\_D. }
When an interruption occurs at LOAD, the newly loaded data are immediately flushed when running the high-level CNN, leading to bandwidth waste.

\textbf{CALC\_I.} 
When an interrupt occurs at CALC\_I, the unsaved final results (generated by previous CALC\_F) should be saved to DDR. The intermediate data from current CALC\_I should also be sent to DDR for further use. At the Recovery stage, the intermediate data should be fetched from DDR. The data movement of intermediate results leads to additional bandwidth requirements.


\textbf{CALC\_F.}
When an interrupt occurs at CALC\_F, there are no intermediate results. 
Although it is necessary to back up the unsaved final results which are generated by previous CALC\_F, these results will be stored in DDR through the subsequent original SAVE instruction.
If the accelerator can record the interrupt status, we can modify the address and workload when executing subsequent original not-virtual save instruction.
In this way, we can avoid the repetitive transmission of the final output results.
% The state records and modifications to normal SAVE instruction will be introduced in the following subsections.
The input data are shared across the CalcBlobs. Thus, the recovery virtual instruction needs to restore the shared input featuremaps.



\textbf{SAVE.}
The overhead of interrupt is only to transfer input data from DDR to the on-chip caches. 

In order to minimize the cost of interrupt, we make the CNN interruptible after the SAVE or CALC\_F. This method only introduces extra data transfer to recovery input data without any extra backup data ($t_2 = 0$). Thus, $t_{cost} = t_4$, in our virtual-instruction-based interrupt.

% Additional virtual instructions also take up bandwidth at instruction fetching phase, even if they are not executed. The instruction number of CALC\_I is tens of times of that of SAVE/CALC\_F. If the network can be interrupted after each CALC\_I, the rapidly increasing virtual instructions reduce the system performance.


\begin{figure*}[t]
	\centering
	\subfloat[$t_1$ for Layer-by-Layer method.]{
		\begin{minipage}[t]{0.45\linewidth}
			\centering
	\includegraphics[width=0.99\linewidth]{fig/t1all.pdf}
		\end{minipage}%
	}
	\subfloat[$t_1$ for Virtual-Instruction method]{
		\begin{minipage}[t]{0.45\linewidth}
			\centering
	\includegraphics[width=0.99\linewidth]{fig/t1after.pdf}
		\end{minipage}%
	}
	\vspace{-1mm} 
	\caption{ An example of waiting time for finishing the current operation ($t_1$) in a convolution layer. Compared with the Layer-by-Layer method, the waiting time of our Virtual-Instruction method is reduced to $1.6\%$ in this example. The reduction in latency is related to the height ($H$) of the input featuremaps. }
	\label{fig:t1example}
\end{figure*}



\subsection {Latency Analysis}
In this subsection, we analyze the impact of interruptible position on the interrupt respond latency.

As introduced in \Cref{sec:instrAcc}, each CALC instruction processes the convolution according to the hardware parallelism with $P_{h}$ lines from $ P_{in} $ input channels to $ P_{out}$ output channels. 
The computation time of each pulse is related to the hardware architecture and the width of the convolution layer. 
The larger the width, the larger the workload of a single calculation instruction, and thus the CALC instruction consumes more time. 
We note the computation of a CALC instruction a $pulse$ and the time consumption of a pulse as a function of the featuremap width, $t_{pulse}$:
\begin{equation}
	t_{pulse}(W) = t_{hw} \times W
\end{equation}
$t_{hw}$ indicates the time for hardware to produce one pixel in the ouput results, which is defined by the architecture and the clock frequency.
$W$ is the width of the featuremaps, indicating the workload of the instruction.


The worst interrupt respond latency in Layer-by-Layer method, is to wait from the beginning of a layer until the whole layer finishes.
The calculation of the whole layer consists of $N_{pulse}^{lbl}$ successive CALC instructions, which is related to the workload ($Ch_{in},Ch_{out},H$) and the hardware parallelism ($P_{in},P_{out},P_{h}$)
\begin{equation}
	\small
	N_{pulse}^{lbl} = \frac{ Ch_{in} \times Ch_{out} \times H }{ P_{in} \times P_{out} \times P_{h} } 
\end{equation}
We note the worst time of waiting to finish the current layer, which is the total time of these pulses, as $t_{1}^{lbl}$.
\begin{equation}
t_{1}^{lbl} = N_{pulse}^{lbl} \times t_{pulse}(W)
\end{equation}
$Ch_{in}$ and $Ch_{out}$ is the number of input channels and output channels. $H$ is the height of featuremaps.

On the other hand, if the execution of the CNN accelerator can be interrupted after SAVE/CALC\_F instructions, the worst case of waiting for finishing the current operation is illustrated in \Cref{fig:t1example}(b). 
The calculation of the whole CalcBlob consists of $N_{pulse}^{VI}$ successive CALC instructions in our Virtual-Instruction method (\textit{VI} method).
\begin{equation}
	\small
	N_{pulse}^{VI} = \frac{ Ch_{in} \times P_{out} \times P_{h} }{ P_{in} \times P_{out} \times P_{h} } 
\end{equation}
We note the worst waiting time of our VI method as $t_{1}^{VI}$.
\begin{equation}
t_{1}^{VI} = N_{pulse}^{VI} \times t_{pulse}(W)
\end{equation}
% Because our VI method only interrupts the execution after CALC\_F and SAVE, there is no extra data transfer for the intermediate results. 
The backup operation in the \textit{VI} method only transfers the final results, which are also transfered to DDR with the SAVE instructions in the Layer-by-Layer method. 
Experimental results, which will be given in \Cref{sec:expt1t2}, show that the data transfer time for the final results is much less than the calculation time (less than 20\%), in both the Layer-by-Layer method ($t_{2}^{lbl}$) and the \textit{VI} method ($t_{2}^{VI}$). 
\begin{equation}
	t_{2}^{lbl} \ll t_{1}^{lbl} ; 
	t_{2}^{VI} \ll t_{1}^{VI}
\end{equation}
Thus the latency to respond to the interrupt request ($t_{latency}$ in \Cref{equ:latency}) is mainly determined by the time of finishing the current operation ($t_1$).
As the interrupt request is unpredictable, we model the interrupt location as evenly distributed within each layer. Thus the average interrupt latency is $\bar{t}_{latency} $.
\begin{equation}
	\bar{t}_{latency} \simeq \frac{1}{2} \times t_{1}
\end{equation}
Compared with the Layer-by-Layer method, the latency of our method is reduced to $R_l$.
\begin{equation}
	\small
	\begin{split}
	R_l & = \frac{\bar{t}_{latency}^{VI}}{\bar{t}_{latency}^{layer}} \simeq \frac{\frac{1}{2} \times t_{1}^{VI}}{\frac{1}{2} \times t_{1}^{layer}} \\
		 & = \frac{ N_{pulse}^{VI} }{N_{pulse}^{lbl} } = \frac{ Ch_{in} \times P_{out} \times P_{h} }{ Ch_{in} \times Ch_{out} \times H } = \frac{ P_{out} \times P_{h} }{ Ch_{out} \times H} 
	\end{split}
	\label{equ:latencyRL}
\end{equation}
$\bar{t}_{latency\_VI}$ and $\bar{t}_{latency\_VI}$ are the average interrupt latency of the Virtual-Instruction method and the Layer-by-Layer method. The effect of latency reduction of the \textit{VI} method is related to the number of output channels ($Ch_{out}$) and featuremap height ($H$). The larger the featuremaps output channels and the height, the better latency reduction result can be achieved.

An example of a convolution layer with a typical size in CNN is given in \Cref{fig:t1example}. The parameters are labelled in the figures ($P_{out}=8,P_{h}=4,Ch_{out}=32,H=60$). 
According to \Cref{equ:latencyRL}, the latency can be reduced to $R_t = (P_{out} \times P_{h}) /( Ch_{out} \times H ) = (8*4)/(32*60) =1.6\%$.
More results will be given in \Cref{sec:exptlatency}.



\subsection{Virtual Instruction ISA (VI-ISA) }
\label{sec:virtualinstr}

We add two fields to the instruction set: 1) Virtual and 2) SaveID, as illustrated in \Cref{fig:instructions}(b). 

\textbf{ Virtual Field}. The virtual instructions should be only valid when interrupt occurs. So we add a field in the original ISA, that indicates whether the instruction is a virtual instruction. If no interrupt occurs, virtual instructions will be skipped and discarded, which can ensure the efficiency of uninterrupted execution. Three values can be set to Virtual Field:
% \begin{itemize}

	\textit{2'b00} indicates this instruction is not virtual, should always be executed.
	
	\textit{2'b01} indicates this instruction is the SAVE instruction for backup. When an interrupt occurs, the high-priority network will start after this instruction.
	
	\textit{2'b10} indicates this instruction is the LOAD instruction for recovery. The corresponding instructions will be executed after the high-priority network.
% \end{itemize}

\textbf{ SaveID Field }
SaveID links CalcBlob instructions to the corresponding SAVE. SaveID of each not-virtual SAVE instruction differs. If the generated outputs of CalcBlobs are stored to DDR by a SAVE instruction, the CalcBlobs have the same SaveID as the SAVE instruction.

% The SaveID for a CalcBlob is the same as its CALC\_F instruction.
One SAVE instruction can correspond to one CalcBlob (Single Blob Save, illustrated in \Cref{fig:singlesave}(b)) or multiple CalcBlobs (Multiple Blob Save, illustrated in \Cref{fig:singlesave}(c)).

For Single Blob Save, no virtual SAVE is added. The high-priority network can be started after the original not-virtual SAVE. The virtual LOAD instructions for data recovery are generated after the original SAVE, and executed after the high-priority network. The execution timeline is shown in \Cref{fig:singlesave}(d).

For Multiple Blob Save, virtual SAVE and LOAD instructions are generated after the CALC\_F of each CalcBlob. When the interrupt request occurs, the virtual SAVE instruction will be executed before the start of the high-priority network. Virtual LOAD instructions for data recovery are executed after the high-priority network. The subsequent original not-virtual SAVE instruction with the same SaveID as the CalcBlob will be modified to avoid duplicate output data transfer. The execution timeline is shown in \Cref{fig:singlesave}(e).


\begin{figure}[t]
	\centering
 % \vspace{-0.1cm} 
 % \setlength{\abovecaptionskip}{0cm} 
 % \setlength{\belowcaptionskip}{-0.4cm} 
	\includegraphics[width=0.99\linewidth]{fig/iau.pdf}
	\vspace{-6mm}
	\caption{Hardware architecture of IAU. The software on the CPU (PS side) communicates with IAU to access the CNN accelerator. IAU records the running state of each task and translates the input instruction virtual instructions sequence (VI-ISA) to a normal sequence of instructions (Original ISA).
	}
	\label{fig:IAU}
\end{figure}

\begin{figure}[t]
	\centering
 % \vspace{-0.1cm} 
 % \setlength{\abovecaptionskip}{0cm} 
 % \setlength{\belowcaptionskip}{-0.4cm} 
	\includegraphics[width=0.8\linewidth]{fig/nestedinter.pdf}
	\vspace{-0mm}
	\caption{
		\textcolor{blue}{The state transition diagram for different task arrangement order.
		}
	}
	\label{fig:nestedinter}
\end{figure}

\begin{figure*}[t]
	\centering
 % \vspace{-0.1cm} 
 % \setlength{\abovecaptionskip}{0cm} 
 % \setlength{\belowcaptionskip}{-0.05cm} 
	\includegraphics[width=0.9\linewidth]{fig/interexample.pdf}
	\vspace{-5mm}
	\caption{ A simple example of our proposed virtual-instruction-based interrupt. Fig(a), the CNN layer structure. Fig(b), the on-chip and DDR addresses of different data. Fig(c), the instruction sequence in virtual instruction ISA (VI-ISA). The blue instructions are the virtual SAVE and LOAD. Fig(d), executed instructions when no interrupt occurs. Fig(e), executed instructions when an interrupt occurs. }
	\label{fig:interexample}
\end{figure*}


\subsection{ Instruction Arrangement Unit (IAU) }

Instruction Arrangement Unit (IAU) is the hardware to handle the computing requirements of different priority tasks. The IAU monitors the interrupt status and generates the original ISA instruction sequence. The original CNN accelerator does not need to know the interrupt status, and only operates the instructions provided by IAU.

The hardware implementation of IAU is shown in \Cref{fig:IAU}, which supports four tasks with different priorities. Task 0 has the highest priority and is not interruptible. 
InstrAddr records the address to fetch the instructions of the corresponding task. The InputOffset and the OutputOffset, which indicate base address offsets of the input and output data, are configured by the software. 
SaveID, SaveAddr, and SaveLength record the status when an interrupt occurs. 
Subsequent not-virtual SAVE instructions will be modified according to the recorded interrupt status (SaveID, SaveAddr, and SaveLength), to avoid duplicate output data transfer.
% An example of the instruction modification will be given at \Cref{sec:exampleVirtual}.

\textcolor{blue}{
It is easy to support nested interrupt use IAU.
There are three kinds of tasks in IAU, the now-running task (NRT), the highest priority task (HPT), and current stalled tasks (CST).
The state transition of two examples are illustrated in \Cref{fig:nestedinter}.
If $NRT=HPT$, the IAU process normally the task, like "Run Task2" at the beginning of \Cref{fig:nestedinter}(a,b).
If a new high-priority task comes, HPT is set to the high-priority, like "Task 0 Comes" at \Cref{fig:nestedinter}(a). Then, IAU backups the running status of NRT, and store the NRT to CST. And then running the high-priority task, like "Run Task 0" in at \Cref{fig:nestedinter}(a).
If a new task with lower priority than NRT. It is directly stored to CST, like "Task 1 Comes" at \Cref{fig:nestedinter}(a).
When NRT finishes, the highest priority task is poped from CST to run/resume, like "Run Taks1" in \Cref{fig:nestedinter}(a) and "Resume Task2" in \Cref{fig:nestedinter}(b).
In \Cref{fig:IAU}, the IAU can support 4 priorities. 
However, there is only one NN task in each priority. 
If we want to support more priorities, the Instruction Arrangement Unit (IAU) needs more hardware resources to expand Status Pool.
Once the number of max supported priorities is determined and programmed on the hardware, it cannot be modified at runtime.
}


\subsection{Example of Virtual Instruction}
\label{sec:exampleVirtual}


The example is based on a straightforward convolutional layer, which has only one input channel and two output channels. 
The convolution kernel size is $1 \times 1$. The shape of the input and output featuremaps is $ 2 \times 16 $ (\Cref{fig:interexample}(a)). The parallelism of the CALC instruction in this example is $ P_{in} = 1$ , $ P_{out}=1$ , $P_{h}=2$.

Thus, the two output channels are calculated by two CALC\_F instructions (instruction 3 and 7 in \Cref{fig:interexample}(c)). The addresses used in the instruction example are listed in \Cref{fig:interexample}(b). \Cref{fig:interexample}(c) is the instruction sequence from DDR with VI-ISA. \Cref{fig:interexample}(d) is the executed original ISA instructions without interrupt. When an interrupt occurs at the first CalcBlob, \Cref{fig:interexample}(e) illustrates the backup/recovery instructions (Blue) and the modified SAVE instruction (Red). 
% IAU monitors the interrupt status and translates the VI-ISA instruction sequence to the original ISA instruction sequence.

% The hardware implementation of IAU is shouwn in \Cref{{fig:IAU}. There is a Status Pool which records the running states of each task. The InstrAddr is records the instruction address of each CNN

\section{Optimization For Post-Processing }
\input{src/hardsoftcodesign.tex}

\section{Evaluation and Results}
\input{src/experiments.tex}

\section{Conclusion}
\label{sec:conclusion}

In this paper, we propose an interruptible CNN accelerator and a deployment framework, INCAME, for multi-robot exploration. 
With the help of the virtual-instruction-based interrupt method, the CNN accelerator can switch between different CNN tasks with low interrupt response latency and low extra cost. 
INCAME only needs to modify the instruction fetch module to IAU in hardware. Thus, it is easy to extend to handle other instruction-driven accelerators.
Therefore, with the help of INCAME, the independent software in ROS can access the accelerator without hardware resources conflicts, on various CNN accelerators.
Note that the development of CPU task scheduling evolved from single-core multi-task to multi-core multi-task. Similarly, INCAME currently focuses on interrupt support for single-core multi-task. We plan to investigate the multi-core multi-tasking for CNN accelerators as part of future work.
INCAME also accelerates the time-consuming post-processing operations like SoftMax, NMS, normalization.
So that the ROS-based MR-Exploration can achieve real-time performance on embedded FPGA. 
% In the future, more robotic algorithms, such as DOpt, Exploration, and Navigation, can be implemented on hardware and included in INCAME, to gain better energy efficiency and achieve better real-time performance. The multi-task supporting multi-core CNN accelerators can also be included in our future work.

% asWith the help of the virtual-instruction based method,

% use section* for acknowledgment
% \section*{Acknowledgment}
% This work is supported by National Key Research and Development Program of China (No. 2017YFA0207600 ), National Natural Science Foundation of China (No. U19B2019, 61832007 ), Tsinghua EE Xilinx AI Research Fund, Beijing National Research Center for Information Science and Technology (BNRist), and Beijing Innovation Center for Future Chips.

% Can use something like this to put references on a page
% by themselves when using endfloat and the captionsoff option.
\ifCLASSOPTIONcaptionsoff
  \newpage
\fi



% trigger a \newpage just before the given reference
% number - used to balance the columns on the last page
% adjust value as needed - may need to be readjusted if
% the document is modified later
%\IEEEtriggeratref{8}
% The "triggered" command can be changed if desired:
%\IEEEtriggercmd{\enlargethispage{-5in}}

% references section

% can use a bibliography generated by BibTeX as a .bbl file
% BibTeX documentation can be easily obtained at:
% http://mirror.ctan.org/biblio/bibtex/contrib/doc/
% The IEEEtran BibTeX style support page is at:
% http://www.michaelshell.org/tex/ieeetran/bibtex/
%\bibliographystyle{IEEEtran}
% argument is your BibTeX string definitions and bibliography database(s)
%\bibliography{IEEEabrv,../bib/paper}
%
% <OR> manually copy in the resultant .bbl file
% set second argument of \begin to the number of references
% (used to reserve space for the reference number labels box)
\bibliographystyle{IEEEtran}
\bibliography{src/fpgaslam}
% biography section
% 
% If you have an EPS/PDF photo (graphicx package needed) extra braces are
% needed around the contents of the optional argument to biography to prevent
% the LaTeX parser from getting confused when it sees the complicated
% \includegraphics command within an optional argument. (You could create
% your own custom macro containing the \includegraphics command to make things
% simpler here.)
%\begin{IEEEbiography}[{\includegraphics[width=0.6in,height=0.8in,clip,keepaspectratio]{mshell}}]{Michael Shell}
% or if you just want to reserve a space for a photo:

% \begin{IEEEbiography}{Michael Shell}
% Biography text here.
% \end{IEEEbiography}

% % if you will not have a photo at all:
% \begin{IEEEbiographynophoto}{John Doe}
% Biography text here.
% \end{IEEEbiographynophoto}

% % insert where needed to balance the two columns on the last page with
% % biographies
% %\newpage

% \begin{IEEEbiographynophoto}{Jane Doe}
% Biography text here.
% \end{IEEEbiographynophoto}


\begin{IEEEbiography}[{\includegraphics[width=0.6in,height=0.8in,clip,keepaspectratio]{fig/yujc.jpg}}]{Jincheng Yu}
  \footnotesize
  received his B.S. degree in electronic engineering department of Tsinghua University, Beijing, China, in 2016. He is currently pursing his Ph.D degree in electronic engineering department of Tsinghua University. His research mainly focues on deep learning acceleration and the hardware architecture for robot systems.
\end{IEEEbiography}



\begin{IEEEbiography}[{\includegraphics[width=0.6in,height=0.8in,clip,keepaspectratio]{fig/xuzhl.jpg}}]{Zhilin Xu}
  \footnotesize
  received his B.S. degree in College of Electronic Engineering from Beijing University of Posts and Telecommunications, Beijing, in 2018. He is currently pursuing his M.S. degree in the Department of Electronic Engineering, Tsinghua University, Beijing. His research mainly focuses on deep learning acceleration and the hardware architecture for robot systems. 
\end{IEEEbiography}


\begin{IEEEbiography}[{\includegraphics[width=0.6in,height=0.8in,clip,keepaspectratio]{fig/zengsl.jpg}}]{Shulin Zeng}
  \footnotesize
  received his B.S. degree in electronic engineering department of Tsinghua University, Beijing, China, in 2014. He is currently pursing his Ph.D degree in electronic engineering department of Tsinghua University. His research mainly focus on software-hardware co-design for deep learning and virtualization in the cloud.
\end{IEEEbiography}


\begin{IEEEbiography}[{\includegraphics[width=0.6in,height=0.8in,clip,keepaspectratio]{fig/yuchao.jpg}}]{Chao Yu}
  \footnotesize
  received her B.S. degree in School of Automation of Beijing Institute of Technology, Beijing, China in 2016. In 2019, she received Master degree in Department of mechanical engineering of Tsinghua University. Now she is pursing the Ph.D degree in Department of Electric Engineering of Tsinghua University. Her research mainly focuses on SLAM, Multi-agent Reinforcement Learning (MARL), and Robotics.
\end{IEEEbiography}

\begin{IEEEbiography}[{\includegraphics[width=0.6in,height=0.8in,clip,keepaspectratio]{fig/qjt.png}}]{Jiantao Qiu}
  \footnotesize
  received the B.S. degree in electronic engineering from Tsinghua University, Beijing, China, in 2015. He is currently pursuing the Ph.D. degree with the Center for Brain-Inspired Computing Research, Tsinghua University, Beijing. His current research interests include computing architecture, reinforcement learning, and system scheduling.
\end{IEEEbiography}


\begin{IEEEbiography}[{\includegraphics[width=0.6in,height=0.8in,clip,keepaspectratio]{fig/scy.jpg}}]{Chaoyang Shen}
  \footnotesize
  is a fourth year student in electronic enginnering deparment of Tsinghua University, Beijing, China. He is going to pursue his bachelor's degree in Tsinghua Shenzhen International Graduate School, Shenzhen, China. His research mainly focuss on  robotic vision.
\end{IEEEbiography}

\begin{IEEEbiography}[{\includegraphics[width=0.6in,height=0.8in,clip,keepaspectratio]{fig/xuyf.jpg}}]{Yuanfan Xu}
  \footnotesize
  is currently pursuing his B.S. degree in electronic engineering in Tsinghua University, Beijing. His research interests include multi-agent reinforcement learning,  the hardware architecture for robot systems and SLAM.
\end{IEEEbiography}

\begin{IEEEbiography}[{\includegraphics[width=0.6in,height=0.8in,clip,keepaspectratio]{fig/dgh.jpg}}]{Guohao Dai}
  \footnotesize
  received his B.S. degree in 2014 and Ph.D degree (with honor) in 2019 from Tsinghua University, Beijing. He is currently a PostDoc Researcher at the Department of Electronic Engineering, Tsinghua University, Beijing. His current research interests include acceleration of large-scale graph processing on hardware and emerging devices, and virtualization techniques in the cloud. He has received Best Paper Award in ASPDAC 2019, and Best Paper Nomination in DATE 2018.
\end{IEEEbiography}

\begin{IEEEbiography}[{\includegraphics[width=0.6in,height=0.8in,clip,keepaspectratio]{fig/wangyu.jpg}}]{Yu Wang}
  \footnotesize
   (S’05-M’07-SM’14) received the BS and   PhD (with honor) degrees from Tsinghua University,   Beijing, in 2002 and 2007. He is currently a tenured   professor with the Department of Electronic Engineering, Tsinghua University. His research interests   include brain inspired computing, application specific hardware computing, parallel circuit analysis,   and power/reliability aware system design methodology. He has authored and coauthored more than 200   papers in refereed journals and conferences. He has   received Best Paper Award in ASPDAC 2019, FPGA   2017, NVMSA 2017, ISVLSI 2012, and Best Poster Award in HEART 2012   with 9 Best Paper Nominations (DATE18, DAC17, ASPDAC16, ASPDAC14,   ASPDAC12, 2 in ASPDAC10, ISLPED09, CODES09). He is a recipient of   DAC under 40 innovator award (2018), IBM X10 Faculty Award (2010).  
  % He served as TPC chair for ICFPT 2019 and 2011, ISVLSI2018, finance   chair of ISLPED 2012-2016, track chair for DATE 2017-2019 and GLSVLSI   2018, and served as program committee member for leading conferences in   these areas, including top EDA conferences such as DAC, DATE, ICCAD,   ASP-DAC, and top FPGA conferences such as FPGA and FPT. 
  Currently,   he serves as co-editor-in-chief of the ACM SIGDA E-Newsletter, associate   editor of the IEEE Transactions on Computer-Aided Design of Integrated   Circuits and Systems,the IEEE Transactions on Circuits and Systems for Video   Technology, the Journal of Circuits, Systems, and Computers,and Special Issue   editor of the Microelectronics Journal. He is now with ACM Distinguished   Speaker Program.   
\end{IEEEbiography}


\begin{IEEEbiography}[{\includegraphics[width=0.6in,height=0.8in,clip,keepaspectratio]{fig/yanghz.png}}]{Huazhong Yang}
  \footnotesize
  (M’97-SM’00-F’20) received B.S.   degree in microelectronics in 1989, M.S. and Ph.D.   degree in electronic engineering in 1993 and 1998,   respectively, all from Tsinghua University, Beijing.   In 1993, he joined the Department of Electronic   Engineering, Tsinghua University, Beijing, where   he has been a Professor since 1998. Prof. Yang   was awarded the Distinguished Young Researcher by   NSFC in 2000, Cheung Kong Scholar by the Chinese   Ministry of Education (CME) in 2012, science and   technology award first prize by China Highway and   Transportation Society in 2016, and technological invention award first prize   by CME in 2019. He has been in charge of several projects, including   projects sponsored by the national science and technology major project, 863   program, NSFC, and several international research projects. Prof. Yang has   authored and co-authored over 500 technical papers, 7 books, and over 180   granted Chinese patents. 
  His current research interests include wireless sensor   networks, data converters, energy-harvesting circuits, nonvolatile processors,   and brain inspired computing. 
  He has also served as the chair of Northern   China ACM SIGDA Chapter science 2014, general co-chair of ASPDAC20,   navigating committee member of AsianHOST18, and TPC member for ASPDAC05, APCCAS06, ICCCAS07, ASQED09, and ICGCS10.  
\end{IEEEbiography}

% You can push biographies down or up by placing
% a \vfill before or after them. The appropriate
% use of \vfill depends on what kind of text is
% on the last page and whether or not the columns
% are being equalized.

%\vfill

% Can be used to pull up biographies so that the bottom of the last one
% is flush with the other column.
%\enlargethispage{-5in}



% that's all folks
\end{document}


