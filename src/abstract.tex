% Multi-Robot Exploration (MR-Exploration) that provides the location and map is the basic task for many multi-robot applications. 
% With the development of Convolutional Neural Network (CNN), the accuracy of some critical components in MR-Exploration, such as Feature-point Extraction (FE) and Place Recognition (PR), can significantly benefit from CNN. 
% To deploy CNN on the embedded real-time system, previous works design CNN accelerators on FPGA. 
% However, these accelerators mainly focus on improving the performance of a single network, lacking support for multi-task. 
% Since researchers in robotic usually run different CNN tasks simultaneously, the inability of accelerators to support multi-task makes it difficult for researchers in robotic to use embedded FPGA. 
% Furthermore, the post-processing of CNN-based components (such as FE and PR), which is also computation consuming, becomes the bottleneck of the system, after accelerating the CNN backbones.
Multi-Robot Exploration (MR-Exploration) that provides the location and map is a primary task for many multi-robot applications. Recent researches introduce Convolutional Neural Network (CNN) into critical components in MR-Exploration, such as Feature-point Extraction (FE) and Place Recognition (PR), to improve the system performance. This CNN-based MR-Exploration needs to run multiple CNN models and complex post-processing algorithms simultaneously, which significantly challenges the hardware platforms of embedded systems.
Previous researches have shown that FPGA is ideal for CNN processing on embedded platforms. However, such accelerators usually process different models in sequence and cannot schedule multiple tasks at runtime. Furthermore, the post-processing of CNNs is also computationally intensive and becomes the bottleneck of the whole system.

To handle such problems, we propose an INterruptible CNN Accelerator for Multi-Robot Exploration (INCAME) framework for the rapid deployment of robot applications on FPGA. In INCAME, we propose an interrupt method based on virtual instructions to support multi-task on CNN accelerators. INCAME also includes hardware modules to accelerate the post-processing of the CNN-based components, organically integrates the post-processing and CNN backbone by sharing memory.
Experimental results show that INCAME enables multi-task scheduling on the CNN accelerator with negligible performance degradation (0.3\%). With the help of multi-task supporting and post-processing acceleration, INCAME enables embedded FPGA to perform MR-Exploration in real time (20 fps).

% To handle such problems, we propose an INterruptible CNN Accelerator for Multi-Robot Exploration (INCAME) framework for rapid deployment of MR-Exploration on FPGA.
% In INCAME, we propose a virtual-instruction-based interrupt method to support multi-task on CNN accelerators.
% INCAME also includes hardware modules to accelerate the post-processing of the CNN-based components.
% % We evaluate INCAME on Xilinx ZU9 MPSoC. 
% The experiment results show that INCAME enables multi-task scheduling on the CNN accelerator with negligible performance reduction (0.8\%). With the help of multi-task supporting and post-processing accelerating, INCAME enables embedded FPGA to execute MR-Exploration in real time (20 fps).
% Multi-Robot Exploration (MR-Exploration) that provides the location and maps is the basic task for many multi-robot applications. 
% Feature-point Extraction (FE) and Place Recognition (PR) are two critical modules in MR-Exploration.
% The accuracy of both modules can benefit from Convolutional Neural Network (CNN).
% Previous CNN accelerators on FPGA mainly focus on improving the performance of a single neural network, lacking multi-task support.
% Researchers in robotic usually run several CNN tasks simultaneously, such as FE and PR.
% The inability of CNN accelerators to support multi-task makes it difficult for researchers in robotic to use embedded FPGA.

% We propose a \textit{MU}lti-\textit{RO}bot \textit{EX}ploration \textit{E}ngine (MUROEXE) to deploy MR-Exploration on embedded FPGA. 
% We propose a virtual-instruction-based interrupt method to support multi-task on CNN accelerators.
% Besides the CNN backbone, the post-precessing for CNN-based FE and PR is also computation consuming. 
% MUROEXE introduces RTL/HLS modules to accelerate the post-precessing of CNN-based modules.
% Experiments show that MUROEXE supports multi-thread scheduling with negligible performance reduction (??\%).
% MUROEXE enables embedded FPGA (Xilinx ZU9) executing MR-Exploration in real-time (30 fps).